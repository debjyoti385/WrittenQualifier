\section*{Overview}
\label{sec:overview}
The term {\em social media} gained attention with the advent of Web 2.0 in the first decade of 20th century \cite{kaplan2010users}. Web 2.0 is also known as {\em Participative} or {\em Social Web} that emphasize on user interaction and user generated content encouraging participatory culture. Before we jump into more details of social media it would be wiser to define it. Though ever evolving social media services makes it hard to define them, most of the research work define it as follows.

\begin{definition}[Social Media]
Social media are interactive computer-mediated technologies that facilitate the creation and sharing of information, ideas, career interests and other forms of expression via virtual communities and networks \cite{kietzmann2011social}.
\end{definition}

In contrast to the {\em traditional media} which operates under a monologic transmission model i.e. one source to many receivers, such as a television, newspaper  or a radio station which broadcasts the same programs to an entire city; {\em social media} are dialogic transmission system which brings interaction, usability and a notion of individual entity in digital world.




%%% Local Variables:
%%% mode: latex
%%% TeX-master: "main"
%%% End:
