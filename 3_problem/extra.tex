\subsection{Calculate surge time series }
To answer $\mathbb{Q}1_{w,[start,end]}$ and $\mathbb{Q}2_{[start,end],\gamma}$,
it is essential to reconstruct the time series of occurrences of events.  PCMS
contains all the information to build surge $s_{t,\tau}$ and then burstiness
$b_{t,\tau}$ time series.  In $\mathbb{Q}1_{w,[start,end]}$ we first
reconstruct surge time series from PCMS with Algorithm \ref{algo:query1approach1}. 

Figure \ref{fig:Q1} shows an example with some entries. As mentioned
earlier the values entered in BurstsPCMS are always greater than 1 which is the
fraction of $c_{t,w}/n_w$. This staircase curve persisted in \emph{BurstsPCMS}
can be transformed to $s_{t,\tau}$ without difficulty. 

According to the definition of $s_{t,\tau}$ in equation \ref{eq:1}, we can
calculate it with $$s_{t,\tau_e} = \alpha(t) s_{t_{i-1}, \tau_e} + (1-
\alphastar) \frac{1}{\tau_e} $$ when we find an occurrences of event $e$ at
$t$.

In \emph{BurstsPCMS} we insert an entry of $w$ only when we find at least $n_w$
occurrences happened in $\tau$ time. Hence, the above equation is used in
calculating surge. It can also be easily noticed that this is a recurrence
relation. We find that this recurrence relation can easily be unwind to
calculate surge for any fraction/times of $n_w$ occurrences happened in $\tau$ time. 

In  \emph{BurstsPCMS} the fraction
$c_{t,w}/n_w$ is exactly the ratio of average inter-event gap for
the $[t-\tau,t]$ interval and lets denote it as $c$. Hence for  $c_{t,w}$
number of occurrences in interval $[t-\tau,t]$; surge $s_{t,\tau_w}$ can be
directly calculated with equation \ref{eq:bulksurge}.  

As mentioned earlier $\alpha(t) = e^{-\frac{t_i-t_{i-1}}{\nu}}$ 
is time variant function. $\alpha(t)$ for $c_{t,w}$ occurrences in $\tau$ interval becomes 

$$\alpha(t) =  e^{-\frac{n_w}{c_{t,w}}} = e^{-\frac{1}{c}}$$ 
It should be noted that $-\frac{1}{c}$ is a constant say $c'$ for the time range $[t,t+\tau]$.
Also $(1-\alphastar) \frac{1}{\tau}$ is also a constant say $k$.  Hence by
expanding the recurrence relation of $s_{t,\tau}$, we get the following which
helps in calculating surge metric easily in Algorithm
\ref{algo:query1approach1}.  


\begin{equation}
\label{eq:bulksurge}
        \begin{split}
                s_{t,\tau} &= e^{c'} s_{t-1,\tau} + k \\
                           &= e^{c'} (e^{c'} s_{t-2,\tau} +k ) + k \\
                           &= e^{2c'} s_{t-2,\tau} + e^{c'}k + k \\
                           &= e^{n_wc'} s_{t-\tau,\tau} + e^{(n_w-1)c'}k + e^{(n_w-2)}k + \ldots + e^{c'}k + k \\
                           &= e^{n_wc'} s_{t-\tau,\tau} + k (e^{(n_w-1)c'} + e^{(n_w-2)c'}+ \ldots + e^{c'} +1)\\
                           &= e^{n_wc'} s_{t-\tau,\tau} + k \frac{1}{1-e^{c'}} ~~~~~~~~~~~~~~~~ [\because  1 < c' < 1 ]\\
                s_{t,\tau} &= e^{-\frac{n_w}{c}} s_{t-\tau,\tau} + (1-\alphastar)\frac{1}{\tau} \frac{1}{1-e^{-\frac{1}{c}}} \\
        \end{split}
\end{equation}

\begin{algorithm}
        \caption{\emph{$\mathbb{Q}1_{w,[start,end]}$ }}
        \label{algo:query1approach1}
\begin{algorithmic} 
\REQUIRE $w, start, end$ where $end > start$ 
\ENSURE $s_{t,\tau}, b_{t,\tau_w}$ time series
\STATE
\STATE Initialize $s_{t,\tau},b_{t,\tau}$ list.
\STATE $lastvalue \leftarrow$ \emph{BurstsPCMS.getCount($w,start$)}
\FOR {$t$ in $start$ to $end$ }
\STATE $val \leftarrow$ \emph{BurstPCMS.getCount(w,t)}
        \IF{ $val > lastvalue$}
        \STATE $interval \leftarrow \tau$
        \STATE \texttt{\# Find if there is any other increase}
        \STATE \texttt{\# in $[t,t+\tau]$ range }
                \IF {\emph{BurstsPCMS.getCount}($w,t+\tau$) $> val$}
                \FOR{$i$ in 1 to $\tau$ }
                \IF{\emph{BurstsPCMS.getCount}($w,t+i,t+\tau$) $> n$ }
                        \STATE $interval \leftarrow i$
                        \STATE break
                        \ENDIF
                \ENDFOR
                \ENDIF
                \STATE $c \leftarrow val - lastvalue$
                \STATE Calculate $s_{t,\tau}$ with $-\frac{1}{c}$ as $c'$ and $interval$ as $n_w$ in equation \ref{eq:bulksurge}.  
                \STATE $lastvalue \leftarrow val$
                \STATE Calculate $b_{t,tau}$ from $s_{t,\tau}$ from interval $[t-\tau,t]$
        \ENDIF
\ENDFOR
\STATE return $s_{t,\tau}$, $b_{t,\tau}$ time series. 
\end{algorithmic}
\end{algorithm}

\begin{figure}[!t]
    \begin{minipage}{0.48\linewidth}
        \centering
        \includegraphics[width=\linewidth]{fig/Q1.png} 
        \vspace{-8mm}
        \caption{$\mathbb{Q}1$ example with staircase curve}\vspace{10mm}
        \label{fig:Q1}
    \end{minipage}
    \hfill\
    \begin{minipage}{0.48\linewidth}
        \centering
        \vspace{-12mm}
        \includegraphics[width=\textwidth]{fig/s_vs_c.eps}
        \vspace{-8mm}
        \caption{Relation between $c$ and $s_{t, \tau}$}
        \label{fig:s_vs_c}
    \end{minipage}
\end{figure}

From example in Figure \ref{fig:Q1}, we can see that there are four increasing
segments within range $[start,end]$. The first segment or staircase $a$ does
not have another increasing segment within $+\tau$ time. Hence the $s_{t,\tau}$
is calculated for the full $\tau$ segment. For the later case segment  $b,c$
is overlapping within $\tau$ time. Hence $s_{t,\tau}$ is calculated for the
segment $b$ till the starting point of segment $c$, and for segment $c$ it is
calculated for $\tau$ time. 


For query $\mathbb{Q}2_{[start,end],\gamma}$, we use heavy-hitter queries. Heavy
hitters query is simpler in this case. Recall that we insert a word into
the PCMS only when it is bursty. Therefore, we can use heavy hitters query of
PCMS to find bursty words within a given range. We can even specify a threshold
that how much times of \emph{burstiness} we want to filter out.

Suppose we want to find the words with burstiness 3 times the normal value. For
3 times $b_{t,\tau}$ the $s_{t,\tau}$ will also be three times. Since we know
that $b_{t,\tau}$ is the area under the curve of $s_{t,\tau}$ for $\tau$ interval
and it only triples iff $s_{t,\tau}$ triples. Hence, we have a direct relation
of $b_{t,\tau}$ with $s_{t,\tau}$. We also claim that there is a relation of $c$
with $s_{t,\tau}$ which helps in estimating $\gamma$ threshold. 

From equation \ref{eq:bulksurge} we can directly plot the relation of
$s_{t,\tau}$ with $c$ while keeping other parameters constant. The plot in
figure \ref{fig:s_vs_c} depicts the relation of threshold $\gamma$ or $c$ with $s_{t,\tau}$. With
a known $\gamma$ the problem statement boils down to retrieve the heavy
hitters.

To be able to efficiently identify the heavy hitters,
we use the dyadic range sum technique of PCMS to decompose
the universe $U$. More precisely, a dyadic range at level $l$
is an interval of the form $[j2^l+1, (j+1)2^l)$ for $l=0, \dots, \log|U|$
and $j=0, \dots, n/2^l-1$. For a particular level $l$, we divide the universe
into $n/2^l$ dyadic ranges, each of size $2^l$, and use a PCMS to
track total frequency of elements in every
dyadic range. Thus we maintain $\log(n+1)$ PCMS in total.
 When an update $(i, c)$ arrives, in each level, we
find the dyadic interval that contains $i$, and update the corresponding
counters.

In order to find all heavy hitters, we ask point queries on the
$\log(n + 1)$ levels recursively. More precisely, suppose at level $l$ we
have identified at most $1/\phi$ heavy dyadic ranges. These ranges are split into $2/\phi$
dyadic ranges in level $l+1$. Then at level $l+1$, we ask point queries on these
$2/\phi$ heavy dyadic ranges to identify again at most $1/\phi$ heavy dyadic
ranges. We repeat this process until we find all heavy hitters at
level 0, which are returned as the estimated heavy hitters.

\begin{algorithm}
        \caption{\emph{$\mathbb{Q}2_{[start,end],\phi}$ }}
        \label{algo:hhapproach1}
\begin{algorithmic} 
\REQUIRE $start, end$ where $end > start$ and threshold $\phi>1$
    \ENSURE $W=\{w_1, \cdots, w_n\}$
\STATE
\STATE return $BurstsPCMS.getHeavyHitters(\phi, start, end)$.
\end{algorithmic}
\end{algorithm}



