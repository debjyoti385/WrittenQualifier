\section{Part B. Ethical concerns:}
\label{part_b}
Artificial intelligence is a boon only if we only use it for the benefit of mankind, same is true with all the technologies we use in our life. Spatio-temporal health analysis as a cumulative measure for public health is a boon for society where government can know the state of public health and act accordingly for the benefit of the state or country and its people. However if the same data is used to target individual person and try to know the concerns related to his/her health that will not be welcome by sane people.

Take another example if the browser history is used to know the health concern of an individual then it is highly unethical and a dangerous proposition. Take a hypothetical situation, say Bluecross BlueShield a health insurance provider for public wants to know the health concerns by analyzing their tweets/ browser history (data collected from 3rd party source). So that they can offer differ pricing on same health insurance plan based on the analysis they have obtained from personal information for monetary gain. The data source used here for such analysis irrespective of its truthfulness makes a judgement on an individual. This information may harm his/her mental, physical, financial health and it is highly unethical with detrimental consequence. As an ethical researcher I have issues in working under such circumstances and in such scenarios.


Taking another hypothtical situation where a government federal agency say {\em Center for Medicare and Medicaid Services} is trying to gauge public health.

 If the cumulative and aggregate metrics on health and ailment is not targeting any individual but a collective society it will be welcome by the reearchers and by me. An example of such scenario; we know social media is very good at first hand reporting of public concerns. If there is an outbreak of a disease in a region then government can respond to the situation by sending doctors and health services team to learn its extent, gather facts and create awareness. Here spatio-temporal analysis just helped to identify concerns of public health; but facts and figures from on ground health workers are mandatory to take any actions like travel advisory and isolation procedures. Spatio-temporal snalysis act as a boon here.

However if the government agency also wants to use personal information to gauge health and habits of an individual then it is unethical and I will keep myself away from it.

%%% Local Variables:
%%% mode: latex
%%% TeX-master: "main"
%%% End:
