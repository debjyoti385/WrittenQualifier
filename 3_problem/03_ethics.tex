\section{Part B. Ethical concerns:}
\label{part_b}
Artificial intelligence is a boon only if we only use it for the benefit of mankind, same is true with all the technologies we use in our life. Spatio-temporal health analysis as a cumulative measure for public health is a boon for the society where government can know the state of public health and act accordingly for the benefit of the state or country and its people. However if the same data is used to target an individual person and gather data related to his/her health with a targeted approach, it will not be welcomed by the general public.Taking example, if the browser history is used to know the health concern of an individual then it is highly unethical and a dangerous proposition.

Here I will highlight some of the key areas of concerns in research with Twitter data:

\vspace{-2mm}
\paragraph{Anonimity:}
Anonymity	 is	 a	 key	 consideration	 in	 research	 ethics,	 particularly	 in	qualitative	 research practices or	when	data	sets	are	shared	outside	of	the	original	research	team. With	traditional	forms	of	research, it	is	generally
straightforward to	anonymise	data	so	that	research	participants	cannot	be	identified.
Further	 problems	 arise	 when	 data	 sets	 are
exported	 to	 external	 coders	 and	 research
partners without anonimizing it \cite{townsend2016social}.

\vspace{-2mm}
\paragraph{Risk of harm:}
The	Association	of	Internet	Researchers	suggest	that
a	researcher's	responsibility	towards	his	or	her	participants	increases	with	the	increased	risk of	 harm	 to	 those	 participants. This includes the risk of using the data to characterize user's health profile and using it against them.


Now take a hypothetical situation, say Bluecross BlueShield a health insurance provider for public wants to know the health concerns of any individual seeking insurance policy by analyzing their tweets/ browser history (data collected from 3rd party source), so that they can offer customized pricing on the same health insurance plan based on the analysis they have obtained from personal information for monetary gain. Irrespective of its truthfulness of the data source used here for such analysis  makes a judgement on an individual.
This information may harm his/her mental, physical, financial health and it is highly unethical with detrimental consequences.
As an ethical researcher, I have issues in working under such circumstances and in such scenarios.

Taking another hypothetical situation where a government federal agency say {\em Center for Medicare and Medicaid Services} is trying to gauge public health.
If the cumulative and aggregate metrics on health and ailment is not targeting any individual but a collective society it will be welcomed by the researchers and by me. An example of such scenario; we know social media is very good at first hand reporting of public concerns. If there is an outbreak of a disease in a region then government can respond to the situation by sending doctors and health services team to learn its extent, gather facts and create awareness. Here spatio-temporal analysis just helped to identify concerns of public health; but facts and figures from on ground health workers are mandatory to take any actions like travel advisory and isolation procedures. Spatio-temporal analysis act as a boon here.

However if the government agency also wants to use personal information to gauge health and habits of an individual without their consent then it is unethical and I would prefer not to be part of this.

\Paragraph{Conclusion:}
The motivation or goal with ethical values of research is a desicive factor for me to join irrespective of the affiliation of research entity.
Ethical decision-making is based on core character values like trustworthiness, respect, responsibility, fairness, caring, and good citizenship.
Honoring ethical values in work is of paramount importance to me.

%%% Local Variables:
%%% mode: latex
%%% TeX-master: "main"
%%% End:
